The next generation network is about convergence of multiple services on to a single multilayered network. Convergence of the network naturally occurs to avoid the need for service specific infrastructures.  However, as convergence occurs, the technology selected for the convergence layer (i.e. WDM, SDH, ATM, MPLS, IP) is influenced by the service mix that a carrier expects to carry in that particular portion of the network.  This leads to different convergence technologies being chosen in different parts of the network.
The selection of different convergence technologies doesn't change the fact that customers are still going to request services that traverse the entire network.  In this context, a service or quality offered by the network is envisaged as a concatenation of dissimilar services offered at multiple switching layers and domains. Consequently, control plane mechanisms must support the routing of service requests through a series of regions using dissimilar convergence layers.  To facilitate this, the control plane needs to understand the multi-layer structure of the network, and how services requested are accommodated.

In the future networks, automated provisioning over multiple network layers will be important for rapid service delivery and cost-effective network operation. Moreover, autonomous recovery mechanisms from network failures should be implemented for rapid network recovery as a lifeline.
Within the scope of GMPLS, three architectural models for the control plane of multilayered networks are defined, the overlay, the augmented or the border model, and the peer model-- a key difference between them being how much and what kind of network information is exchanged between individual layers.

In this thesis, I propose to investigate several novel schemes � including a novel aggregation technique, and a number of multilayer path selection policies� suitable for the implementation in multilayered and multi-domain hierarchical networks. The proposed schemes address the dynamic provisioning of service constrained and/or failure-disjoint LSPs that cross vertical as well as horizontal layers of the network hierarchy in two settings: a) an intra-carrier, and b) inter-carrier scenarios. This process will enable inter-layer resource optimization (e.g. between optical and packet layers) for both the primary and backup traffic capacity.
