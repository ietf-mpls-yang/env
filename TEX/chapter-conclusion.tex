\chapter{Conclusions and Future Research}
% new right header
\markright{Conclusions and Future Research}

\section{Concluding Remarks}
We conclude this thesis proposal with a review of the addressed problems and the proposed solutions, as well as the tools that will be used to solve them. We highlight the work that has been completed and what still needs to be done to conclude the remaining part:

\begin{itemize}

\item Review of the design and planning of horizontally and vertically layered transport networks�we have conducted a broad study on how multiple switching layer interaction impacts the network design and planning, and its effect on the dynamic LSP provisioning paradigms (e,g, independent single-layer provisioning, or integrated multi-layer provisioning). We considered the challenges posed in providing the appropriate level of protection � at each individual layer� for a connection according to a predefined service availability requirement.

\item Novel hierarchical aggregation scheme� we proposed a scheme that is suitable for the integration in the border model multilayered multi-domain network architecture. The scheme defines the notion of link resource trees that encode useful information about links at each of the network switching layers (e.g. SRLG, and layer-specific QoS information). The aggregated scheme is extended to carry summarized information about abstract links between domain border nodes.

\item Intra-carrier inter-domain hierarchical QoS path calculation/selection policy� we presented a path selection scheme suitable for deployment in a multilayered transport network that makes use of the hierarchical aggregation scheme mentioned above. The scheme is capable of the computation of optimal service guaranteed and/or failure diverse LSPs across multiple switching layers and domains.

\item Inter-carrier inter-domain QoS path calculation/selection policy� we addressed an important and timely issue in today�s multi-provider networks to provide multi-carrier service guaranteed SLAs across heterogeneous networks in the presence of minimal or no exchange of network state information. To solve this problem, we propose 2 potential resolutions: 1) a game theory approach to optimize the coordination and partitioning of the overall service among all transiting domains especially when ISPs exercise heterogeneous service-cost models, and 2) a discovery message flooding mechanism to collect the service pledge on per domain for each of the transit domains. The receiver in turn, makes the most favorable decision on path selection.

\item Inter-layer coordination protection schemes� we reviewed several techniques for the coordination of the restoration procedures at different switching layers, including recovery at the highest layer, lowest layer, or timer and/or token coordinated. In our study, we will attempt to realize a suitable technique that provides efficient cross-layer and cross domain coordination for end-to-end recovery.

\end{itemize}

\section{Future Research}

This is the future research...
